\documentclass[11pt]{article}
\usepackage{graphicx}
\usepackage{caption}
\usepackage{geometry}
\usepackage[colorlinks=true, linkcolor=blue, urlcolor=blue]{hyperref}
\geometry{margin=1in}

\title{Abilene Historical Zoning Workflow}
\author{}
\date{\today}

\begin{document}
\maketitle

\section*{Summary}
I am piloting a VLM-centric parsing pipeline for historical newspapers, inspired by
Reducto’s vision that spatially aware models can outperform traditional OCR in noisy
layouts. This Abilene sheet serves as the proving ground: I cleaned the scan, assigned
stable glyph identifiers, probed LLM reading-order reasoning, and reconciled the zoning
inset with modern TIGER shapefiles and satellite context. The sections below walk through
those stages and the artifacts they produced.

\section*{Workflow}
\subsection*{Stage 1: Original Radiance Sheet}
We begin with the raw scan and perform light normalization only to preserve every stroke
that will later inform both contour extraction and glyph segmentation (Figure~\ref{fig:original}).

\begin{figure}[ht]
    \centering
    \includegraphics[width=\textwidth]{../src/scratchpad/agentic_outputs/abilene/original.png}
    \caption{Stage 1 --- Original radiance scan of the Abilene zoning sheet before any processing.}
    \label{fig:original}
\end{figure}

\subsection*{Stage 2: Global Contours (Counters)}
Running a deskew-and-threshold pass yields a binary field where I can trace contours of
the underlying zoning map. The resulting contour field captures line geometry, not text
regions, and serves as the backbone for later alignment to TIGER data (Figure~\ref{fig:contours}).

\begin{figure}[ht]
    \centering
    \includegraphics[width=\textwidth]{../src/scratchpad/agentic_outputs/abilene/original_contours.png}
    \caption{Stage 2 --- Global contour detection (\emph{aka} counters) across the full sheet.}
    \label{fig:contours}
\end{figure}

\subsection*{Stage 3: Numbered Bounding Boxes \& Reading Order}
Separately, we run connected-component analysis on the binarized glyph layer to carve out
axis-aligned bounding boxes for every text fragment. Unlike the contours above (which hug
strokes), these boxes are derived from grouped glyph pixels and produce clean regions that
can be indexed. The enumeration proceeds left-to-right within each horizontal band, and
my inspection shows the numbering is already reliable—the ambiguity lies in knowing when
to stop scanning a row, hop to an inset, or resume mid-column. To probe that uncertainty,
I fed the box coordinates plus cropped thumbnails into an LLM, asking it to emit the
reading order and justify each transition. The long-term plan is to mirror Reducto’s
chunk-and-compose strategy: let a VLM partition the sheet into semantically coherent
panels, run localized extraction on each chunk, and then stitch the outputs back together
using the glyph numbering as the spine so that the final sequence reflects how a human
would navigate the page (Figures~\ref{fig:numbered} and~\ref{fig:numbered-detail}).

\begin{figure}[ht]
    \centering
    \includegraphics[width=\textwidth]{../src/scratchpad/agentic_outputs/abilene/numbered/page_glyphs_numbered.png}
    \caption{Stage 3 --- All bounding boxes numbered so each glyph region can be referenced downstream.}
    \label{fig:numbered}
\end{figure}

\begin{figure}[ht]
    \centering
    \includegraphics[width=0.75\textwidth]{page_glyphs_numbered_zoom.png}
    \caption{Stage 3 (detail) --- Zoomed section of the numbered page showing the exact crop sent to the VLM with coordinates.}
    \label{fig:numbered-detail}
\end{figure}

\subsection*{Stage 4: Zoning Inset Contours}
With glyph IDs locked in, we manually crop to the zoning inset and re-run the contour pass
so the geometry within the map panel is isolated from the rest of the sheet (Figure~\ref{fig:map-contours}).
In principle a VLM could pick these cut points automatically, which is part of the future
direction for reducing human-in-the-loop steps.

\begin{figure}[ht]
    \centering
    \includegraphics[width=\textwidth]{../src/scratchpad/agentic_outputs/abilene/zoning_map_original_lowerleft/zoning_contours.png}
    \caption{Stage 4 --- Contours from the extracted zoning map inset after cropping.}
    \label{fig:map-contours}
\end{figure}

\subsection*{Stage 5: Contours Meets Shapefile}
The isolated contours are intersected with the TIGER shapefile outline to check scale,
orientation, and topology before committing to a full overlay. The first pass quickly
surfaced the obvious blockers: the historic scan lacks projection control points, the scale
is off, and the inset is skewed relative to modern north-up geometry (Figure~\ref{fig:contours-shapefile}).

\begin{figure}[ht]
    \centering
    \includegraphics[width=\textwidth]{../src/scratchpad/agentic_outputs/abilene/zoning_map_original_lowerleft/contours_plus_shapefile.png}
    \caption{Stage 5 --- Historical contours overlain on the shapefile outline to validate the geometry.}
    \label{fig:contours-shapefile}
\end{figure}

\subsection*{Stage 6: Shapefile on Original Map}
After verifying the inset alignment, we cast the shapefile edges back over the original
sheet to check global positioning and ensure that the inset registration carries over to
the full composition (Figure~\ref{fig:map-shapefile}).

\begin{figure}[ht]
    \centering
    \includegraphics[width=\textwidth]{../src/scratchpad/agentic_outputs/abilene/zoning_map_original_lowerleft/abilene_outline_overlay.png}
    \caption{Stage 6 --- Shapefile outline reprojected directly on the original map for a full-page check.}
    \label{fig:map-shapefile}
\end{figure}

\subsection*{Stage 7: Satellite Validation}
Finally, we blend the red historical overlay and shapefile edges with a satellite basemap
to confirm that the digitized geometry lands where modern Abilene exists today (Figure~\ref{fig:satellite}).

\begin{figure}[ht]
    \centering
    \includegraphics[width=\textwidth]{../src/scratchpad/agentic_outputs/abilene/zoning_map_original_lowerleft/satellite_with_zoning_red.png}
    \caption{Stage 7 --- Satellite basemap with the red historical overlay and shapefile edges to verify present-day alignment.}
    \label{fig:satellite}
\end{figure}

\end{document}
