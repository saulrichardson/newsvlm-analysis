\documentclass[11pt]{article}

\usepackage[margin=1in]{geometry}
\usepackage{booktabs}
\usepackage{tabularx}
\usepackage{longtable}
\usepackage{graphicx}
\usepackage{xcolor}
\usepackage{hyperref}
\usepackage{enumitem}
\usepackage{caption}
\usepackage{float}

\hypersetup{hidelinks}

% Generated by scripts/build_hybrid_regulatory_topics_latex_report.py
\IfFileExists{provenance.tex}{% Auto-generated provenance.
\newcommand{\ReportDate}{2026-02-03 01:11:40 PST}
\newcommand{\HybridRunDir}{\texttt{\detokenize{/Users/saulrichardson/projects/newspaper-parsing/tmp_hybrid_pi_real_1}}}
\newcommand{\HybridClausesPath}{\texttt{\detokenize{/Users/saulrichardson/projects/newspaper-parsing/tmp_hybrid_pi_real_1/clause_out/clauses.jsonl}}}
\newcommand{\HybridClustersPath}{\texttt{\detokenize{/Users/saulrichardson/projects/newspaper-parsing/tmp_hybrid_pi_real_1/clause_clusters_local/clusters.jsonl}}}
}{}

\title{Hybrid Regulatory Topic Discovery: From Chunk Topics to Clause-Level Instruments}
\author{}
\date{\IfFileExists{provenance.tex}{\ReportDate}{(run builder to populate provenance)}}

\begin{document}
\maketitle

\section{Problem and goal}

PI feedback motivating this work:
\begin{quote}
\emph{The pure clustering approach is interesting but isn’t surfacing regulatory topics or questions to a sufficiently granular degree.}
\end{quote}

The key design change is to shift the clustering target from \textbf{whole chunks} (which can mix multiple themes)
to \textbf{atomic clauses/requirements} (which are closer to ``regulatory instruments'').

\section{Baseline: chunk-level topic clustering (existing pipeline)}

The repository already contains a mature chunk-topic discovery pipeline
(provider embeddings $\rightarrow$ UMAP $\rightarrow$ HDBSCAN $\rightarrow$ LLM labels).

\IfFileExists{figures/baseline_full_ordinance_umap.png}{
\begin{figure}[H]
  \centering
  \includegraphics[width=0.95\textwidth]{figures/baseline_full_ordinance_umap.png}
  \caption{Baseline example: chunk-level UMAP + HDBSCAN scatter (full ordinances). Each point is a chunk, so mixed-content chunks limit topic specificity.}
\end{figure}
}{}

\IfFileExists{tables/baseline_top_clusters.tex}{\begin{table}[H]
\centering
\caption{Baseline chunk-topic clusters (top by weight; full ordinances).}
\label{tab:baseline}
\begin{tabularx}{\textwidth}{rrrX}
\toprule
cluster & weight & size & topic label \\
\midrule
84 & 228.58 & 2790 & Public legal notices and real estate sale advertisements \\
119 & 72.93 & 1638 & Definitions and variance rules in city zoning code \\
109 & 58.31 & 400 & Adoption of zoning map, plat, and building-permit rules \\
19 & 53.79 & 533 & Rezoning map changes with public hearing to reclassify parcels \\
9 & 51.64 & 813 & Mobile home park regulations: site plans, setbacks, permits \\
176 & 45.74 & 468 & Non-zoning public notices and international news briefs \\
167 & 36.17 & 416 & Public bidding and hearing notices for contracts and probate \\
116 & 33.90 & 375 & Real estate classifieds and property transfer notices \\
82 & 28.54 & 698 & Parking space standards and loading requirements \\
36 & 27.81 & 543 & Housing health standards and occupancy licensing enforcement \\
110 & 27.61 & 261 & Rezoning and district map amendments with plats and historic districts \\
7 & 26.02 & 309 & Public hearing for conditional use permit in R-1 zone \\
\bottomrule
\end{tabularx}
\vspace{0.25em}
\par\small\textit{These clusters are over multi-thousand-character chunks; hybrid clustering instead targets extracted atomic requirements.}
\end{table}
}{}

\section{Hybrid pipeline (implemented)}

The implemented hybrid workflow is documented in:
\begin{itemize}[leftmargin=*]
  \item \texttt{docs/hybrid\_regulatory\_topics\_workflow.md}
\end{itemize}

At a high level:
\begin{enumerate}[leftmargin=*]
  \item Segment each document into coherent sections (optional but PI-recommended).
  \item Label each section with a \emph{primary regulatory motive} (controlled taxonomy).
  \item Extract \emph{atomic clauses/requirements} per section (LLM; strict schema when run via OpenAI Batch).
  \item Embed + cluster extracted clauses to discover granular instruments.
  \item Label clause clusters into human-readable instruments (LLM).
\end{enumerate}

\subsection{Concrete artifacts (scripts and prompts)}

The pipeline above corresponds to concrete repo artifacts:
\begin{itemize}[leftmargin=*]
  \item Section segmentation:
    \texttt{scripts/export\_regulatory\_section\_segmentation\_batch\_requests.py},
    \texttt{scripts/rehydrate\_regulatory\_section\_segmentation\_openai\_batch\_results.py},
    prompt \texttt{prompts/regulatory\_section\_segmenter\_prompt\_text.txt}.
  \item Motive labeling:
    \texttt{scripts/export\_regulatory\_motive\_batch\_requests.py},
    \texttt{scripts/rehydrate\_regulatory\_motive\_openai\_batch\_results.py},
    prompt \texttt{prompts/regulatory\_motive\_classifier\_prompt\_text.txt}.
  \item Clause extraction:
    \texttt{scripts/export\_regulatory\_clause\_extraction\_batch\_requests.py},
    \texttt{scripts/rehydrate\_regulatory\_clause\_extraction\_openai\_batch\_results.py},
    prompt \texttt{prompts/regulatory\_clause\_extractor\_prompt\_text.txt}.
  \item Clause clustering:
    local \texttt{scripts/cluster\_regulatory\_clauses\_local.py} (TF--IDF $\rightarrow$ SVD $\rightarrow$ UMAP $\rightarrow$ HDBSCAN),
    or embedding-based \texttt{scripts/export\_clause\_embedding\_batch\_requests.py} +
    \texttt{scripts/rehydrate\_clause\_embeddings\_openai\_batch\_results.py} +
    \texttt{scripts/cluster\_clause\_embeddings.py}.
  \item Cluster $\rightarrow$ instrument labeling:
    \texttt{scripts/export\_regulatory\_instrument\_cluster\_labeling\_batch\_requests.py},
    \texttt{scripts/rehydrate\_regulatory\_instrument\_cluster\_labels\_openai\_batch\_results.py},
    prompt \texttt{prompts/regulatory\_instrument\_cluster\_labeler\_prompt\_text.txt}.
  \item Human inspection packet:
    \texttt{scripts/build\_regulatory\_instrument\_cluster\_report.py}.
  \item RHS-style panels from instrument clusters:
    \texttt{scripts/build\_regression\_rhs\_from\_instrument\_clusters.py}.
\end{itemize}

\section{Smoke test: real ordinance sample}

This report is grounded in a concrete run directory:
\begin{itemize}[leftmargin=*]
  \item Hybrid run: \HybridRunDir
\end{itemize}

\subsection{Section segmentation}

\IfFileExists{tables/sections_index.tex}{\begin{table}[H]
\centering
\caption{LLM-driven section segmentation (excerpt).}
\label{tab:sections}
\begin{tabularx}{\textwidth}{lrXr}
\toprule
doc\_id & section & title & chars \\
\midrule
ad-visor-and-chronicle\_\_2010-05-08 & 1 & Intro article: Safe Blade Safety Cover & 2194 \\
ad-visor-and-chronicle\_\_2010-05-08 & 2 & Ordinance text: Home Occupations (First copy) & 2560 \\
ad-visor-and-chronicle\_\_2010-05-08 & 3 & Advertisements \& Notices & 1124 \\
ad-visor-and-chronicle\_\_2010-05-08 & 4 & Ordinance text: Home Occupations (Second copy) & 5225 \\
ad-visor-and-chronicle\_\_2010-05-08 & 5 & Advertisements \& Notices (Continuation) & 344 \\
akron-register-tribune\_\_1974-09-26 & 1 & Preamble, Interpretation \& DEFINITIONS (Zoning Ordinance) & 54861 \\
akron-register-tribune\_\_1974-09-26 & 2 & Parking, Access, Loading \& General Regulations (Zoning Overview) & 12092 \\
akron-register-tribune\_\_1974-09-26 & 3 & Solid Waste Ordinance \& City Proceedings (Appendix \& Notices) & 6047 \\
albert-lea-evening-tribune\_\_1961-10-24 & 1 & Intro / Ordinance Title & 31409 \\
albert-lea-evening-tribune\_\_1961-10-24 & 2 & Definitions & 8819 \\
albert-lea-evening-tribune\_\_1961-10-24 & 3 & Establishment of Districts & 3542 \\
albert-lea-evening-tribune\_\_1961-10-24 & 4 & Regulations for RA Districts and Specific Districts & 6968 \\
albert-lea-evening-tribune\_\_1961-10-24 & 5 & Board of Zoning Administration; Administration, Permits, Fees, Enforcement, Public Hearings & 32195 \\
\bottomrule
\end{tabularx}
\vspace{0.25em}
\par\small\textit{This is the key coherence fix: articles/ads/notices are separated from ordinance text before motive labeling + clause extraction.}
\end{table}
}{}

\subsection{Motive labeling summary}

\IfFileExists{tables/chunk_motive_counts.tex}{\begin{table}[H]
\centering
\caption{Chunk/section primary motive counts (hybrid run).}
\label{tab:chunk_motives}
\begin{tabularx}{\textwidth}{Xr}
\toprule
motive & n\_chunks \\
\midrule
other & 3 \\
infrastructure\_coordination & 3 \\
public\_health\_safety & 2 \\
unclear & 2 \\
externality\_control & 2 \\
growth\_management & 1 \\
\bottomrule
\end{tabularx}
\vspace{0.25em}
\par\small\textit{These are section-level motive labels (controlled taxonomy) used to guide clause extraction and allow motive-filtered clustering.}
\end{table}
}{}

\subsection{Clause extraction summary}

\IfFileExists{tables/clause_motive_counts.tex}{\begin{table}[H]
\centering
\caption{Extracted clause counts by motive (hybrid run).}
\label{tab:clause_motives}
\begin{tabularx}{\textwidth}{Xr}
\toprule
motive & n\_clauses \\
\midrule
infrastructure\_coordination & 52 \\
growth\_management & 22 \\
environmental\_protection & 14 \\
public\_health\_safety & 11 \\
unclear & 7 \\
aesthetic\_design\_control & 4 \\
externality\_control & 2 \\
exclusion & 1 \\
\bottomrule
\end{tabularx}
\vspace{0.25em}
\par\small\textit{Clauses are the units being embedded/clusted, not entire chunks. This is the main granularity change.}
\end{table}
}{}
\IfFileExists{tables/clause_modality_counts.tex}{\begin{table}[H]
\centering
\caption{Extracted clause counts by modality (hybrid run).}
\label{tab:clause_modalities}
\begin{tabularx}{\textwidth}{Xr}
\toprule
modality & n\_clauses \\
\midrule
must & 69 \\
must\_not & 21 \\
may & 14 \\
definition & 9 \\
\bottomrule
\end{tabularx}
\vspace{0.25em}
\par\small\textit{A sanity check: we expect many 'must'/'must\_not' requirements; definition clauses can be filtered if they overwhelm instruments.}
\end{table}
}{}

\subsection{Clause clustering (instrument discovery)}

\IfFileExists{figures/hybrid_clause_umap.png}{
\begin{figure}[H]
  \centering
  \includegraphics[width=0.95\textwidth]{figures/hybrid_clause_umap.png}
  \caption{Hybrid example: clause-level UMAP + HDBSCAN scatter. Each point is a single extracted requirement, which enables instrument-level clusters.}
\end{figure}
}{}

\IfFileExists{tables/cluster_index.tex}{\begin{table}[H]
\centering
\caption{Clause-level instrument clusters (top by size).}
\label{tab:clusters}
\begin{tabularx}{\textwidth}{rrrXXXX}
\toprule
cluster & n\_clauses & n\_docs & instrument & top motives & top modalities & keywords \\
\midrule
1 & 15 & 2 & Parking space requirements by use & infrastructure\_coordination:9; growth\_management:4; environmental\_protection:2 & must:13; must\_not:2 & space, ft, parking space, plot, sq, sq ft, parking, plus, building, area \\
8 & 11 & 1 & Residential Home Occupation Restrictions & public\_health\_safety:6; aesthetic\_design\_control:2; exclusion:1 & must\_not:6; must:4 & home, home occupation, occupation, premises, shall, residential, commercial, occupations, home occupations, produced \\
9 & 10 & 1 & Dumpster disposal restrictions & environmental\_protection:6; infrastructure\_coordination:3; public\_health\_safety:1 & must\_not:8; must:1 & disposal, shall acceptable, acceptable disposal, acceptable, shall, waste, waste shall, materials, prohibited, allowed \\
0 & 7 & 1 & Parking stall requirements and maintenance & infrastructure\_coordination:7 & must:7 & stalls, parking stalls, required, stalls required, parking, driveways, automobile, unit, guest, beds \\
2 & 7 & 1 & Permitted RA uses & infrastructure\_coordination:6; growth\_management:1 & may:6; must\_not:1 & permitted, uses, permitted ra, include, ra, uses include, ra uses, schools, operated, commercial \\
4 & 7 & 1 & Town Board governance and rules & unclear:4; growth\_management:2; infrastructure\_coordination:1 & must:6; may:1 & board, town board, town, times, rules, ordinance, act, evidence, provide, time \\
6 & 7 & 2 & Parking space requirements & infrastructure\_coordination:7 & must:6; definition:1 & spaces, parking spaces, parking, dwellings, dwelling unit, unit, bedroom dwelling, bedroom, unit parking, dwelling \\
7 & 6 & 1 & Non-conforming use regulations & infrastructure\_coordination:4; growth\_management:1; aesthetic\_design\_control:1 & may:3; must:2 & non, non conforming, conforming, conforming use, use, amendment, building, provided, conform, land \\
3 & 5 & 2 & Official Design and Conformity & infrastructure\_coordination:3; growth\_management:2 & must:5 & design, official, official design, regulations, structure, district, adoption, conform, requirements, parking requirements \\
5 & 5 & 2 & Off-street loading and parking standards & infrastructure\_coordination:4; growth\_management:1 & must:5 & feet, street, loading, 25, 25 feet, spaces, driveways, employee, 300, alley \\
\bottomrule
\end{tabularx}
\vspace{0.25em}
\par\small\textit{Each row is a discovered instrument-like cluster over extracted clauses (not chunks).}
\end{table}
}{}

\subsection{Example instrument clusters (verbatim evidence)}

\IfFileExists{snippets/cluster_examples.tex}{% Auto-generated cluster examples (verbatim evidence + paraphrase).

\subsubsection*{Cluster 1: Parking space requirements by use}
\noindent\textbf{Instrument description:} This cluster collects minimum parking-space requirements tied to land-use categories. Parking rules are typically calculated by capacity (seats, beds, employees) or floor area, and apply to uses such as churches, hotels, hospitals, community facilities, boarding houses, home occupations, and other commercial/industrial uses. One clause also addresses yard/open space demarcation to avoid double counting.

\noindent\textbf{Keywords (TF-IDF):} space, ft, parking space, plot, sq, sq ft, parking, plus, building, area, floor, floor area

\begin{itemize}[leftmargin=*]
\item
\textbf{Requirement:} Parking space required: one space per four seats or bench seating capacity.

\begin{quote}\footnotesize Auditoriums, assembly halls, dance halls, theaters, gymnasiums, and skating rinks: One space for each four seats or bench seating capacity.\end{quote}
\noindent\texttt{doc=akron-register-tribune\_\_1974-09-26 | chunk=akron-register-tribune\_\_1974-09-26::sec003 | motive=infrastructure\_coordination | modality=must}
\item
\textbf{Requirement:} One parking space per sleeping room for boarding/rooming houses.

\begin{quote}\footnotesize Boarding, rooming or lodging house: One space for each sleeping room.\end{quote}
\noindent\texttt{doc=akron-register-tribune\_\_1974-09-26 | chunk=akron-register-tribune\_\_1974-09-26::sec003 | motive=infrastructure\_coordination | modality=must}
\end{itemize}

\subsubsection*{Cluster 5: Off-street loading and parking standards}
\noindent\textbf{Instrument description:} Regulates off-street loading/unloading spaces, including access from streets or alleys and all-weather surface requirements. Also sets parking-related dimensions and location rules, such as driveway width (12–25 feet) and stalls within 300 feet on premises (employees up to 1,000 feet). It additionally requires dwelling-unit abutment of 25 feet along a public street.

\noindent\textbf{Keywords (TF-IDF):} feet, street, loading, 25, 25 feet, spaces, driveways, employee, 300, alley, access, street alley

\begin{itemize}[leftmargin=*]
\item
\textbf{Requirement:} Loading/unloading space with access from a street or alley and 14 feet vertical clearance must be provided.

\begin{quote}\footnotesize loading and unloading space with proper access from a street or alley, and with at least fourteen (14) feet of vertical clearance shall be provided, either within or outside the building to adequately serve the use on the lot.\end{quote}
\noindent\texttt{doc=akron-register-tribune\_\_1974-09-26 | chunk=akron-register-tribune\_\_1974-09-26::sec003 | motive=infrastructure\_coordination | modality=must}
\item
\textbf{Requirement:} All off-street loading spaces must have an all-weather surface.

\begin{quote}\footnotesize All off-street loading or unloading spaces shall have an all-weather surface to provide safe and convenient access and use during all seasons.\end{quote}
\noindent\texttt{doc=akron-register-tribune\_\_1974-09-26 | chunk=akron-register-tribune\_\_1974-09-26::sec003 | motive=infrastructure\_coordination | modality=must}
\end{itemize}

\subsubsection*{Cluster 8: Residential Home Occupation Restrictions}
\noindent\textbf{Instrument description:} Regulates home-based businesses in residential areas by restricting sales to items produced by the home occupation, prohibiting commercial-vehicle deliveries, and limiting activity to within the dwelling or permanent accessory structure. The occupation must be incidental to residential use, with limited staffing, no exterior alterations or changes to residential appearance, and no adverse impact on neighbors’ health, safety, or neighborhood character.

\noindent\textbf{Keywords (TF-IDF):} home, home occupation, occupation, premises, shall, residential, commercial, occupations, home occupations, produced, produced home, vehicle

\begin{itemize}[leftmargin=*]
\item
\textbf{Requirement:} No on-premises selling of articles or services unless they are produced by the home occupation.

\begin{quote}\footnotesize D) No article or service shall be sold or offered for sale on the premises, except such as is produced by such occupation.\end{quote}
\noindent\texttt{doc=ad-visor-and-chronicle\_\_2010-05-08 | chunk=ad-visor-and-chronicle\_\_2010-05-08::sec002 | motive=public\_health\_safety | modality=must\_not}
\item
\textbf{Requirement:} Home occupations must not use commercial vehicles for delivery to or from the premises; a commercial vehicle is defined by signage or other advertisement of the home occupation.

\begin{quote}\footnotesize H) A home occupation shall not involve the use of commercial vehicles for delivery of materials to or from the premises. For the purposes of this section, a commercial vehicle shall be defined as one with any sign, markings, address, telephone number, or other form of display that advertises or is associated with a home occupation on that premises.\end{quote}
\noindent\texttt{doc=ad-visor-and-chronicle\_\_2010-05-08 | chunk=ad-visor-and-chronicle\_\_2010-05-08::sec002 | motive=public\_health\_safety | modality=must\_not}
\end{itemize}

\subsubsection*{Cluster 9: Dumpster disposal restrictions}
\noindent\textbf{Instrument description:} This cluster enforces dumpster-based waste disposal rules, specifying acceptable materials and service access. It includes first-come, first-served operation, prohibitions on liquids, hazardous waste, Freon appliances, yard waste, and household garbage, plus conditions on dumpster availability and delivery restrictions.

\noindent\textbf{Keywords (TF-IDF):} disposal, shall acceptable, acceptable disposal, acceptable, shall, waste, waste shall, materials, prohibited, allowed, dumpster, service

\begin{itemize}[leftmargin=*]
\item
\textbf{Requirement:} The disposal event operates on a first-come, first-served basis.

\begin{quote}\footnotesize This is on first come, first service.\end{quote}
\noindent\texttt{doc=ad-visor-and-chronicle\_\_2010-05-08 | chunk=ad-visor-and-chronicle\_\_2010-05-08::sec003 | motive=infrastructure\_coordination | modality=must}
\item
\textbf{Requirement:} Liquid waste (paint, oil, chemicals, etc.) shall not be acceptable for disposal.

\begin{quote}\footnotesize Liquid Waste (paint, oil, Chemicals, Etc.)\end{quote}
\noindent\texttt{doc=ad-visor-and-chronicle\_\_2010-05-08 | chunk=ad-visor-and-chronicle\_\_2010-05-08::sec003 | motive=environmental\_protection | modality=must\_not}
\end{itemize}

\subsubsection*{Cluster 7: Non-conforming use regulations}
\noindent\textbf{Instrument description:} This cluster governs non-conforming uses, including how they may be extended within buildings and changed within the same structure, with limits on extending beyond the building and the effect of demolition on status. It also covers amortization or conformance timelines for non-conforming signs and allowances for ordinary repairs that do not increase the building's cubic content.

\noindent\textbf{Keywords (TF-IDF):} non, non conforming, conforming, conforming use, use, amendment, building, provided, conform, land, signs, structure

\begin{itemize}[leftmargin=*]
\item
\textbf{Requirement:} Non-conforming use may be extended within the parts of the building designed for such use (as of the time of adoption or amendment).

\begin{quote}\footnotesize Any non-conforming use may be extended throughout any of the parts of a building which were manifestly arranged or designed for such use at the time of adoption or amendment of this Ordinance,\end{quote}
\noindent\texttt{doc=akron-register-tribune\_\_1974-09-26 | chunk=akron-register-tribune\_\_1974-09-26::sec003 | motive=infrastructure\_coordination | modality=may}
\item
\textbf{Requirement:} Non-conforming use may not extend beyond the building.

\begin{quote}\footnotesize but no such use shall be extended to occupy any land outside such building.\end{quote}
\noindent\texttt{doc=akron-register-tribune\_\_1974-09-26 | chunk=akron-register-tribune\_\_1974-09-26::sec003 | motive=infrastructure\_coordination | modality=must\_not}
\end{itemize}
}{}

\subsection{Document-level purposes (LLM)}

\IfFileExists{tables/doc_purposes.tex}{\begin{table}[H]
\centering
\caption{Document-level purpose summaries (LLM; excerpt).}
\label{tab:purposes}
\begin{tabularx}{\textwidth}{lXX}
\toprule
doc\_id & purposes (motive: description) & notes \\
\midrule
ad-visor-and-chronicle\_\_2010-05-08 & externality\_control: Regulate potential nuisance externalities from home occupations (noise, odors, unsightly conditions) and restrict activities that could affect neighboring properties (storage, deliveries, signage). | public\_health\_safety: Protect health, safety, and welfare of residents by restricting factors that could create hazards or health concerns in the neighborhood. | aesthetic\_design\_control: Preserve residential appearance and minimize visual impact by limiting exterior changes and signage related to home occupations. | infrastructure\_coordination: Limit use of traffic-generating activities (e.g., deliveries by commercial vehicles) to reduce impact on local infrastructure and traffic. | exclusion: Restrict commercial scope to prevent off-site sales and ensure activities stay subordinate to residential use. & Document analyzed is Marshall City Zoning Code amendment (home occupations). OCR text also includes unrelated article/ads about a blade-safety device; analysis focused on the zoning ordinance provisions and their regulatory aims. \\
akron-register-tribune\_\_1974-09-26 & public\_health\_safety: Promote the health and safety of the City’s residents (including fire safety and protection from dangers). | growth\_management: Lessen street congestion and avoid undue concentration of population to guide urban growth. | infrastructure\_coordination: Facilitate the adequate provision of transportation, water, sewerage, schools, parks, and other public requirements. | externality\_control: Conserve the value of buildings and encourage the most appropriate use of land throughout the municipality (targeting land-use externalities). | other: Moral considerations mentioned in the preamble; not clearly mapped to other motives in the taxonomy. & Preamble establishes broad, multi-faceted regulatory aims (health/safety, growth and congestion management, infrastructure provisioning, and land-value preservation). No explicit separate motives beyond general welfare; some morals mention is mapped to 'other' due to taxonomy limits. \\
albert-lea-evening-tribune\_\_1961-10-24 & public\_health\_safety: Regulates districts and permitted uses to safeguard public health and safety and to promote the public welfare (health, safety, order, convenience and general welfare) as a primary objective of the zoning regime. | externality\_control: Controls nuisance-causing emissions and environmental externalities by prohibiting or restricting uses that generate odors, dust, smoke, gas, vibration or noise. | growth\_management: Imposes quantitative and design limits (density/plot area, height, yards, lot coverage) to steer growth and land-use patterns. | infrastructure\_coordination: Regulates parking, street use and access to coordinate transportation networks and utilities with land-use planning. | exclusion: Prohibits or limits certain trades and industries to exclude undesirable activities from certain districts. | aesthetic\_design\_control: Imposes setback and building-coverage rules to shape the visual character and compatibility of development. & Document is a multi-district zoning ordinance. Main regulatory aims are visible in health/safety provisions, nuisance/externality controls, growth/density limits, parking/infrastructure rules, and explicit prohibitions. Some items overlap across categories (e.g., environmental externalities vs. environmental protection). \\
\bottomrule
\end{tabularx}
\vspace{0.25em}
\par\small\textit{Purpose extraction is useful for document-level framing; clause clustering is the main instrument discovery step.}
\end{table}
}{}

\section{Why this improves granularity}

Chunk clustering often groups together:
\begin{itemize}[leftmargin=*]
  \item multiple regulatory instruments inside one ordinance section, and/or
  \item ordinance text mixed with publication notices, hearings, meeting minutes, ads, etc.
\end{itemize}

The hybrid pipeline clusters \textbf{atomic requirements}, which makes it easier to surface:
\begin{itemize}[leftmargin=*]
  \item distinct instruments inside the same ordinance (e.g., parking minimums by use vs.\ loading standards),
  \item stable, interpretable ``RHS-style'' covariates built from instrument clusters.
\end{itemize}

\section{Known limitations and next knobs}

\begin{itemize}[leftmargin=*]
  \item Some clusters are administrative/procedural (e.g., governance rules). Decide whether to keep or filter them.
  \item Local gateway runs may not enforce OpenAI-side JSON-schema output; prefer Batch for large runs.
  \item For regression use, cluster IDs are run-specific; treat a clustering run as a frozen artifact or build a stable mapping layer.
\end{itemize}

\end{document}
