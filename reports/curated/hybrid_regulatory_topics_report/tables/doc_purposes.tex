\begin{table}[H]
\centering
\caption{Document-level purpose summaries (LLM; excerpt).}
\label{tab:purposes}
\begin{tabularx}{\textwidth}{lXX}
\toprule
doc\_id & purposes (motive: description) & notes \\
\midrule
ad-visor-and-chronicle\_\_2010-05-08 & externality\_control: Regulate potential nuisance externalities from home occupations (noise, odors, unsightly conditions) and restrict activities that could affect neighboring properties (storage, deliveries, signage). | public\_health\_safety: Protect health, safety, and welfare of residents by restricting factors that could create hazards or health concerns in the neighborhood. | aesthetic\_design\_control: Preserve residential appearance and minimize visual impact by limiting exterior changes and signage related to home occupations. | infrastructure\_coordination: Limit use of traffic-generating activities (e.g., deliveries by commercial vehicles) to reduce impact on local infrastructure and traffic. | exclusion: Restrict commercial scope to prevent off-site sales and ensure activities stay subordinate to residential use. & Document analyzed is Marshall City Zoning Code amendment (home occupations). OCR text also includes unrelated article/ads about a blade-safety device; analysis focused on the zoning ordinance provisions and their regulatory aims. \\
akron-register-tribune\_\_1974-09-26 & public\_health\_safety: Promote the health and safety of the City’s residents (including fire safety and protection from dangers). | growth\_management: Lessen street congestion and avoid undue concentration of population to guide urban growth. | infrastructure\_coordination: Facilitate the adequate provision of transportation, water, sewerage, schools, parks, and other public requirements. | externality\_control: Conserve the value of buildings and encourage the most appropriate use of land throughout the municipality (targeting land-use externalities). | other: Moral considerations mentioned in the preamble; not clearly mapped to other motives in the taxonomy. & Preamble establishes broad, multi-faceted regulatory aims (health/safety, growth and congestion management, infrastructure provisioning, and land-value preservation). No explicit separate motives beyond general welfare; some morals mention is mapped to 'other' due to taxonomy limits. \\
albert-lea-evening-tribune\_\_1961-10-24 & public\_health\_safety: Regulates districts and permitted uses to safeguard public health and safety and to promote the public welfare (health, safety, order, convenience and general welfare) as a primary objective of the zoning regime. | externality\_control: Controls nuisance-causing emissions and environmental externalities by prohibiting or restricting uses that generate odors, dust, smoke, gas, vibration or noise. | growth\_management: Imposes quantitative and design limits (density/plot area, height, yards, lot coverage) to steer growth and land-use patterns. | infrastructure\_coordination: Regulates parking, street use and access to coordinate transportation networks and utilities with land-use planning. | exclusion: Prohibits or limits certain trades and industries to exclude undesirable activities from certain districts. | aesthetic\_design\_control: Imposes setback and building-coverage rules to shape the visual character and compatibility of development. & Document is a multi-district zoning ordinance. Main regulatory aims are visible in health/safety provisions, nuisance/externality controls, growth/density limits, parking/infrastructure rules, and explicit prohibitions. Some items overlap across categories (e.g., environmental externalities vs. environmental protection). \\
\bottomrule
\end{tabularx}
\vspace{0.25em}
\par\small\textit{Purpose extraction is useful for document-level framing; clause clustering is the main instrument discovery step.}
\end{table}
